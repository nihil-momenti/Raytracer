\section{Background}
  \subsection{Ray tracing}
    Ray tracing is a very high quality but performance intensive method of
    generating 3D images.  Ray tracing can be implemented in two fashions,
    `forwards' or `light-based' ray tracing works by firing rays out of lights
    and tracing their path throught the scene until they coincide with the view
    point. This is computationally infeasible as close to 0\% of the rays fired
    from the lights actually coincide with the view point.  This form of ray
    tracing is used in some algorithms such as photon mapping, in this case
    instead of tracing each ray until they intersect the eye they are just
    traced until a cut-off point is reached and the intermediate intersections
    of the ray are used to generate the image.

    The other form of ray tracing is `backwards' or 'eye-based' ray tracing,
    this works by tracing rays from the eye out into the scene and calculating
    the colour at each ray's intersection with the scene.  Initial ray casting
    algorithms terminated at this point with just a simple check for whether to
    light source is occluded by another object at this point or not.  Newer
    algorithms instead compute reflections and refractions from this point as
    well. This is done by tracing new rays from the intersection, these rays
    directions are determined by the shape and material properties at the
    intersection point.

  \subsection{Shapes}
    For the following section t.........:
    \begin{quote}\begin{itemize}
      \item Lowercase unbolded refers to scalars: $t$
      \item Capitals refer to points: $P$
      \item Bold refers to vectors: $\mathbf{v}$
      \item If a point and vector have the same letter the vector is the
        position vector of the point: $P \Leftrightarrow \mathbf{p}$
    \end{itemize}\end{quote}

    \noindent The equation used for rays is:
    \[ P = P_i + t \mathbf{v} \]
    Where $P$ is any point on the ray, $P_i$ is a point that the ray passes
    through, $t$ is a scalar and $\mathbf{v}$ is the direction vector for the
    ray.

    \subsubsection{Plane}
      One of the simplest shapes to implement in ray tracing is a plane.  We
      start with the point normal form of the plane equation:
      \[ \left( \mathbf{p} - \mathbf{p}_0 \right) \cdot \mathbf{n} = 0 \]
      This describes a plane that passes through the point $\mathbf{p}_0$ with
      normal $\mathbf{n}$.  Any point $\mathbf{p}$ which satisfies this equation
      lies within the plane.

      Equating this with the equation for a ray we can solve for $t$, the
      distance along the ray that it intersects the plane:
      \[ t  = \frac{\left( \mathbf{p}_0 - \mathbf{p}_i \right) \cdot
         \mathbf{n}}{\mathbf{v} \cdot \mathbf{n}} \]

      Finding the normal to the plane is even easier, at all points it is just
      the normal.

    \subsubsection{Sphere}
      The next easiest shape to implement is the sphere.  This is described by
      the equation:
      \[ \left| \mathbf{p} - \mathbf{p}_0 \right| = r \]
      Where $\mathbf{p}_0$ is the center of the sphere, $r$ is the radius of the
      sphere and $\mathbf{p}$ is any point on the surface of the sphere.

      Again we can equate this with the equation for a ray to determine where a
      ray intersects the sphere:
      \[ t = \mathbf{v} \cdot \mathbf{q} \pm \sqrt{\left(\mathbf{v} \cdot
         \mathbf{q}\right)^2 + 2 \left(\mathbf{q} \cdot \mathbf{q} - r^2\right)}
         \]
      
      The normal for a sphere is slightly more difficult than the plane:
      \[ \mathbf{n} = \frac{\mathbf{p} - \mathbf{p}_0}{\left| \mathbf{p} -
         \mathbf{p}_0 \right|} \]
