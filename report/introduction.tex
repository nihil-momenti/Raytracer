\chapter{Introduction}
  This project was based around creating a prototype ray tracer for generating
  3d imagery.  There were two stages to implementing this ray tracer, the basic
  ray tracer and some extensions of the ray tracer.
  
  \vspace{10pt}

  The basic ray tracer is required to:
  \begin{quote}\begin{itemize}
    \item Be able to generate scenes including planes and spheres with a
      single directional light source and an ambient light.
    \item Generate shadows from the directional light when objects occlude each
      other from the light
    \item Generate a view from any point looking at any other point with a
      provided field-of-view
    \item Include at least one reflective sphere and must include at least one
      planar surface with a pattern of "texture" on it.
  \end{itemize}\end{quote}

  \vspace{10pt}

  The extensions that were implemented included:
  \begin{quote}\begin{itemize}
    \item Multiple light sources.  As well as being able to add multiple light
      sources to the scene 4 different types of light sources were provided.
    \item Anti-aliasing. Simple multi-sample anti-aliasing was implemented, this
      had some optimization applied to it.
    \item Transformations. Simple transformations of objects in the scene were
      implemented.
    \item Constructive solid geometry (CSG). Only intersection based CSG was
      implemented, the data model chosen for represent shape-ray intersections
      proved to not work with union CSG.
  \end{itemize}\end{quote}
  
    
  % stating the context, goals and limitations of the project.

